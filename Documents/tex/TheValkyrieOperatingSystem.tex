% SPDX-License-Identifier: CC-BY-SA-4.0
\documentclass[14pt,openany]{article}

\usepackage{graphicx}
\usepackage[utf8]{inputenc}
\usepackage{fancyhdr}
\usepackage{titling}

\pagestyle{fancy}
\fancyhf{}
\cfoot{\thepage}

\pretitle{\begin{center}\Huge\bfseries\vspace*{\fill}}
\posttitle{\par\vskip 1em
  \includegraphics[width=0.5\textwidth]{../www/assets/valkyrie.png}\par
  \vspace*{\fill}\end{center}}

\title{The Valkyrie Operating System}
\author{Vincent4486}
\date{2025}

\begin{document}
\maketitle
\newpage
\tableofcontents\footnote{Java and Java Virtual Machine are trademarks of Oracle and/or its affiliates.}
\newpage

\section{Introduction}
The Valkyrie Operating System is an OS designed to run on the x86 architecture. In this operating system, the Java Virtual Machine

\newpage
\section{Building}

The Valkyrie Operating System uses SCons as its build system and requires a cross-toolchain for x86 targets. The build process is designed for Linux hosts.

\subsection{Prerequisites}

\subsubsection{Supported Host Environments}
\begin{itemize}
\item \textbf{Linux} distributions (Debian/Ubuntu, Fedora, Arch, openSUSE, Alpine).
\end{itemize}

\subsubsection{Required Build Artifacts}
\begin{itemize}
\item binutils: 2.45
\item gcc: 15.2.0
\item musl (for sysroot): 1.2.5
\end{itemize}

\subsubsection{Installing Dependencies}

On Linux, use the provided script to install packages:

\begin{verbatim}
sudo ./scripts/base/dependencies.sh
\end{verbatim}

This script detects your distribution and installs the necessary packages, such as:
\begin{itemize}
\item Debian/Ubuntu: \texttt{libmpfr-dev libgmp-dev libmpc-dev gcc python3 scons python3-sh dosfstools}
\item Fedora: \texttt{mpfr-devel gmp-devel libmpc-devel gcc python3 scons python3-sh dosfstools}
\item Arch: \texttt{mpfr gmp mpc gcc python scons python-sh dosfstools}
\end{itemize}

\subsection{Build Steps}

1) \textbf{(Optional) Install distribution packages on Linux:}

\begin{verbatim}
sudo ./scripts/base/dependencies.sh
\end{verbatim}

2) \textbf{Build a cross-toolchain (recommended):}

Use the SCons phony target:

\begin{verbatim}
scons toolchain
\end{verbatim}

This creates a \texttt{toolchain/} directory with subdirectories like \texttt{toolchain/i686-elf}.

Alternatively, run the helper script directly:

\begin{verbatim}
./scripts/base/toolchain.sh toolchain i686-elf
\end{verbatim}

3) \textbf{Build the project:}

\begin{verbatim}
scons
\end{verbatim}

Common build options:
\begin{itemize}
\item \texttt{config=debug|release} - Build configuration (default: debug)
\item \texttt{arch=i686|x64} - Target architecture (default: i686)
\item \texttt{imageFS=fat12|fat16|fat32|ext2} - Filesystem for image (default: fat32)
\item \texttt{imageSize=250m} - Image size (supports k/m/g)
\item \texttt{outputFile=<name>} - Base name for output image (default: image)
\item \texttt{buildType=full|kernel|usr|image} - What to build (default: full)
\end{itemize}

Examples:

\begin{verbatim}
scons config=release
scons arch=x64 config=release
scons buildType=kernel  # build only the kernel
\end{verbatim}

\subsection{Running and Debugging}

The SConstruct file provides phony targets for common tasks:

\begin{itemize}
\item \texttt{scons run} — Run the built image under QEMU
\item \texttt{scons debug} — Start a GDB-enabled debug session
\item \texttt{scons bochs} — Run with Bochs
\item \texttt{scons toolchain} — (Re)build the cross-toolchain
\end{itemize}

Or call helper scripts directly:

\begin{verbatim}
./scripts/base/qemu.sh disk build/i686_debug/valkyrix
./scripts/base/gdb.sh disk build/i686_debug/valkyrix
\end{verbatim}

\subsection{Notes and Tips}

\begin{itemize}
\item The build is designed for Linux hosts; full image creation tools are Linux-focused.
\item Building the toolchain can take significant time and requires build dependencies.
\item If you have a prebuilt cross-toolchain, place it in \texttt{toolchain/<target>} and SCons will use it.
\end{itemize}

\newpage
\section{Code}
\end{document}
